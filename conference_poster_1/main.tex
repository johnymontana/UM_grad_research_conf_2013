%%%%%%%%%%%%%%%%%%%%%%%%%%%%%%%%%%%%%%%%%
% Jacobs Landscape Poster
% LaTeX Template
% Version 1.0 (29/03/13)
%
% Created by:
% Computational Physics and Biophysics Group, Jacobs University
% https://teamwork.jacobs-university.de:8443/confluence/display/CoPandBiG/LaTeX+Poster
% 
% Further modified by:
% Nathaniel Johnston (nathaniel@njohnston.ca)
%
% This template has been downloaded from:
% http://www.LaTeXTemplates.com
%
% License:
% CC BY-NC-SA 3.0 (http://creativecommons.org/licenses/by-nc-sa/3.0/)
%
%%%%%%%%%%%%%%%%%%%%%%%%%%%%%%%%%%%%%%%%%

%----------------------------------------------------------------------------------------
%	PACKAGES AND OTHER DOCUMENT CONFIGURATIONS
%----------------------------------------------------------------------------------------

\documentclass[final]{beamer}

\usepackage[scale=1.24]{beamerposter} % Use the beamerposter package for laying out the poster

\usetheme{confposter} % Use the confposter theme supplied with this template

\setbeamercolor{block title}{fg=ngreen,bg=white} % Colors of the block titles
\setbeamercolor{block body}{fg=black,bg=white} % Colors of the body of blocks
\setbeamercolor{block alerted title}{fg=white,bg=dblue!70} % Colors of the highlighted block titles
\setbeamercolor{block alerted body}{fg=black,bg=dblue!10} % Colors of the body of highlighted blocks
% Many more colors are available for use in beamerthemeconfposter.sty

%-----------------------------------------------------------
% Define the column widths and overall poster size
% To set effective sepwid, onecolwid and twocolwid values, first choose how many columns you want and how much separation you want between columns
% In this template, the separation width chosen is 0.024 of the paper width and a 4-column layout
% onecolwid should therefore be (1-(# of columns+1)*sepwid)/# of columns e.g. (1-(4+1)*0.024)/4 = 0.22
% Set twocolwid to be (2*onecolwid)+sepwid = 0.464
% Set threecolwid to be (3*onecolwid)+2*sepwid = 0.708

\newlength{\sepwid}
\newlength{\onecolwid}
\newlength{\twocolwid}
\newlength{\threecolwid}
\setlength{\paperwidth}{48in} % A0 width: 46.8in
\setlength{\paperheight}{36in} % A0 height: 33.1in
\setlength{\sepwid}{0.024\paperwidth} % Separation width (white space) between columns
\setlength{\onecolwid}{0.22\paperwidth} % Width of one column
\setlength{\twocolwid}{0.464\paperwidth} % Width of two columns
\setlength{\threecolwid}{0.708\paperwidth} % Width of three columns
\setlength{\topmargin}{-0.5in} % Reduce the top margin size
%-----------------------------------------------------------

\usepackage{graphicx}  % Required for including images

\usepackage{booktabs} % Top and bottom rules for tables

%----------------------------------------------------------------------------------------
%	TITLE SECTION 
%----------------------------------------------------------------------------------------

\title{WilliViz - A Data Visualization Framework For iOS} % Poster title

\author{William Lyon} % Author(s)

\institute{Department of Computer Science, University of Montana} % Institution(s)

%----------------------------------------------------------------------------------------

\begin{document}

\addtobeamertemplate{block end}{}{\vspace*{2ex}} % White space under blocks
\addtobeamertemplate{block alerted end}{}{\vspace*{2ex}} % White space under highlighted (alert) blocks

\setlength{\belowcaptionskip}{2ex} % White space under figures
\setlength\belowdisplayshortskip{2ex} % White space under equations

\begin{frame}[t] % The whole poster is enclosed in one beamer frame

\begin{columns}[t] % The whole poster consists of three major columns, the second of which is split into two columns twice - the [t] option aligns each column's content to the top

\begin{column}{\sepwid}\end{column} % Empty spacer column

\begin{column}{\onecolwid} % The first column

%----------------------------------------------------------------------------------------
%	OBJECTIVES
%----------------------------------------------------------------------------------------

\begin{alertblock}{Objectives}

Objective header here. 
\begin{itemize}
\item Develop a framework for data visualization on iOS devices
\item Identify data visualization techniques/principles that are specifically suited for mobile devices.
\item Framework should be generic enough to allow developers to easily integrate with their apps  
\item Develop a portfolio of data visualizations
\item Develop a tutorial / documentation to allow developers to learn how to use WilliViz
\end{itemize}

\end{alertblock}

%----------------------------------------------------------------------------------------
%	INTRODUCTION
%----------------------------------------------------------------------------------------

\begin{block}{Introduction}

We live in a time of Big Data. We have access to more data than ever before. We also have mobile devices that we cary with us at all times. Visualizing this data is an important step in analyzing and making use of this abundant information. Mobile devices present a unique and different platform for data visualization. The nature of the device and the manner in which we use them influences how we can use mobile devices as a platform for data visualization. WilliViz attempts to leverage some of these mobile-platform specific characteritics and provide a toolkit to allow other software developers to create beautiful, rich, interactive data visualizations for iOS devices.

\end{block}

%------------------------------------------------

% scree shot here???
\begin{figure}
\includegraphics[width=0.8\linewidth]{placeholder.jpg}
\caption{Figure caption}
\end{figure}

%----------------------------------------------------------------------------------------

\end{column} % End of the first column

\begin{column}{\sepwid}\end{column} % Empty spacer column

\begin{column}{\twocolwid} % Begin a column which is two columns wide (column 2)

\begin{columns}[t,totalwidth=\twocolwid] % Split up the two columns wide column

\begin{column}{\onecolwid}\vspace{-.6in} % The first column within column 2 (column 2.1)

%----------------------------------------------------------------------------------------
%	MATERIALS
%----------------------------------------------------------------------------------------

\begin{block}{Data Visualization Paradigms For Mobile}

The following help describe the paradigm of data visualization for mobile devices:
% research this a bit: UI design for mobile / data viz in general
% citations!
\begin{enumerate}
\item Identify data visualizations to be implemented
\item Delegate / Protocol software design pattern used to allow easy integration for developers
\item Duis porta consequat lorem
\item Curabitur pellentesque dignissim
\end{enumerate}

\end{block}

%----------------------------------------------------------------------------------------

\end{column} % End of column 2.1

\begin{column}{\onecolwid}\vspace{-.6in} % The second column within column 2 (column 2.2)

%----------------------------------------------------------------------------------------
%	METHODS
%----------------------------------------------------------------------------------------

\begin{block}{Implementation}
\begin{enumerate}
\item Objective-C Programming Language
\item CoreGraphics drawing library
\item Delegate / protocol design pattern
\item Scaling coordiante system based on data
\item Built on top of Apple's UIView class
\item Plots are drawn independently, multiple plots on sceen at one time
\end{enumerate}
\end{block}

%---------------------------------------------------------------------------re -------------

\end{column} % End of column 2.2

\end{columns} % End of the split of column 2 - any content after this will now take up 2 columns width

%----------------------------------------------------------------------------------------
%	IMPORTANT RESULT
%----------------------------------------------------------------------------------------

\begin{alertblock}{Code Example} % code listings package???

-(void)drawWilliViz:<WilliVizDataSourceDelegate>id delete;

[setDataSourceDelegate: self];

\end{alertblock} 

%----------------------------------------------------------------------------------------

\begin{columns}[t,totalwidth=\twocolwid] % Split up the two columns wide column again

\begin{column}{\onecolwid} % The first column within column 2 (column 2.1)

%----------------------------------------------------------------------------------------
%	MATHEMATICAL SECTION
%----------------------------------------------------------------------------------------

\begin{block}{Code Section / How to use} % flowchart? how a developer would use to integrate into their app

FLOWCHART HERE

\end{block}

%----------------------------------------------------------------------------------------

\end{column} % End of column 2.1

\begin{column}{\onecolwid} % The second column within column 2 (column 2.2)

%----------------------------------------------------------------------------------------
%	RESULTS
%----------------------------------------------------------------------------------------

\begin{block}{Examples}

\begin{figure}
\includegraphics[width=0.8\linewidth]{placeholder.jpg}
\caption{Figure caption}
\end{figure}

\begin{figure}
\includegraphics[width=0.8\linewidth]{placeholder.jpg}
\caption{Figure caption}
\end{figure}



\end{block}

%----------------------------------------------------------------------------------------

\end{column} % End of column 2.2

\end{columns} % End of the split of column 2

\end{column} % End of the second column

\begin{column}{\sepwid}\end{column} % Empty spacer column

\begin{column}{\onecolwid} % The third column

%----------------------------------------------------------------------------------------
%	CONCLUSION
%----------------------------------------------------------------------------------------

\begin{block}{Conclusion}

Nunc tempus venenatis facilisis. \textbf{Curabitur suscipit} consequat eros non porttitor. Sed a massa dolor, id ornare enim. Fusce quis massa dictum tortor \textbf{tincidunt mattis}. Donec quam est, lobortis quis pretium at, laoreet scelerisque lacus. Nam quis odio enim, in molestie libero. Vivamus cursus mi at \textit{nulla elementum sollicitudin}.

\end{block}

%----------------------------------------------------------------------------------------
%	ADDITIONAL INFORMATION
%----------------------------------------------------------------------------------------

\begin{block}{Future Work}

Future things to do with WilliViz
\begin{itemize}
\item Developer documentation
\item Distribution - open source?
\item App showcase / portfolio
\item Data collection
\item User interactiveness
\end{itemize}

\end{block}

%----------------------------------------------------------------------------------------
%	REFERENCES
%----------------------------------------------------------------------------------------

\begin{block}{References}

\nocite{*} % Insert publications even if they are not cited in the poster
\small{\bibliographystyle{unsrt}
\begin{itemize}
	\item Stanford CS193p course
	\item Apple CoreGraphics developer documentation
	\item WWDC video
	\item Carneige Mellon data visualization for iPad course
\end{itemize}}
%\bibliography{sample}\vspace{0.75in}}

\end{block}

%----------------------------------------------------------------------------------------
%	ACKNOWLEDGEMENTS
%----------------------------------------------------------------------------------------

\setbeamercolor{block title}{fg=red,bg=white} % Change the block title color

\begin{block}{Acknowledgements}

\small{\rmfamily{This research is the product of a Special Topics course facilitated by Professor Jesse Johnson. The purpose of which was to identify a Massive Open Online Course (MOOC) for study.  Special thanks to Professor Jesse Johnson for facilitating this course and encouraging me to work on this project.}} \\

\end{block}

%----------------------------------------------------------------------------------------
%	CONTACT INFORMATION
%----------------------------------------------------------------------------------------

\setbeamercolor{block alerted title}{fg=black,bg=norange} % Change the alert block title colors
\setbeamercolor{block alerted body}{fg=black,bg=white} % Change the alert block body colors

\begin{alertblock}{Contact Information}

\begin{itemize}
\item Web: \href{http://www.university.edu/smithlab}{lyonwj.com}
\item Email: \href{mailto:lyonwj@gmail.com}{lyonwj@gmail.com}
\item Twitter: \href{http://www.twitter.com/lyonwj}{@lyonwj}
\end{itemize}

\end{alertblock}

\begin{center}
\begin{tabular}{ccc}
\includegraphics[width=0.4\linewidth]{logo.png} & \hfill & \includegraphics[width=0.4\linewidth]{logo.png}
\end{tabular}
\end{center}

%----------------------------------------------------------------------------------------

\end{column} % End of the third column

\end{columns} % End of all the columns in the poster

\end{frame} % End of the enclosing frame

\end{document}
